\chapter{Test e risultati}
	\begin{figure}[!bp]
		\centering
		\includegraphics[width=10cm]{img/bln.png}
	\end{figure}
	Nella fase di implementazione di questo progetto mi sono servito di un'immagine rappresentata da una lista di 100 file .grd raffiguranti le altezze del terreno di un'area geografica percorsa dal fiume secchia.

	I test condotti su \texttt{bln\_reader} vengono utilizzati per controllare il correto funzionamento del programma.\\
	Dati in input la lista di file relativi alle tavolette e uno rappresentante un poligono, in modalit\`{a} debug il vengono stampati su terminale le informazioni interessanti che vanno a comporre \texttt{map\_info.txt} e la griglia di tavolette, ovvero una matrice 10x10, in cui viene evidenziata l'area del bounding box. \\
	Sono stati testati 6 diversi poligoni e da cui sono stati generati 6 diversi bounding box e 6 diversi file .PTS.
	

	Per quanto riguarda \texttt{multires}, invece, per ogni bln 