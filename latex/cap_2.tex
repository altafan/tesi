\chapter{Obiettivi}

	\section{Stato dell'arte}
	Nella versione originale del codice vengono per prima cosa memorizzati in due strutture dati le informazioni contenute nei file relativi all'immagine e al poligono. La matrice relativa all'immagine viene memorizzata in un array monodimensionale, mentre viene utilizzata una \texttt{struct} formata da due vettori per contenere le coordinate dei vertici del poligono e le condizioni di bordo.

	In secondo luogo, operando sulla matrice dell'immagine, ad ogni cella viene assegnato un valore che indica se il corrispondente punto in coordinare reali \`{e} interno o esterno al poligono. Vengono, infine, calcolate le condizioni al contorno dei punti sul perimetro del poligono.

	Oltre all'informazione della posizione rispetto al poligono, alle celle della matrice vengono anche assegnati i valori delle altezze del terreno e, solo in seguito, viene creata la matrice a multi risoluzione, in cui questi valori vengono copiati. 

	Il problema legato a questo tipo di approccio sta nel fatto che la matrice a multi risoluzione viene creata sulla base delle dimensioni della matrice che rappresenta la mappa altimetrica. \\
	Anche se le dimensioni della matrice a multi risoluzione sono inferiori fra le 10 e le 50 volte rispetto a quelle dell'immagine, \`{e} comunque necessario molto spazio in memoria per poterla allocare, date le grandi dimensioni dell'immagine.
	\section{Obiettivo della tesi}
	L'obiettivo di questa tesi \`{e}, dunque, quello di riutilizzare il codice originale in modo tale da creare una nuova versione del programma in cui venga ridotto l'impatto in memoria della matrice a multi risoluzione, nella quale i valori delle altezze sono caricate direttamente dai file solo dopo che questa \`{e} stata creata.
	\section{Soluzioni adottate}
	Per ovviare al problema dell'allocazione di memoria, nella nuova versione del programma, operando sulla matrice di tavolette, viene identificata la sottomatrice di punti che contiene il poligono, che viene chiamata \textbf{bounding box}.\\
	Creando la matrice a multi risoluzione sulla base delle dimensioni del bounding box viene ridotto lo spazio di memoria necessario alla sua allocazione. 
	La matrice relativa all'immagine viene utilizzata come bitmap della matrice a multi risoluzione. In questa matrice vengono suddivisi i punti fra interni ed esterni e vengono poi sistemate le condizioni al contorno. I blocchi a diversi livelli di risoluzione contenuti nella matrice vengono generati da un set di punti di partenza rappresentanti il perimetro del poligono.

	Nel capitolo successivo viene descritto come questa soluzione \`{e} stata realizzata e viene confrontata con quella precedente mostrando le principali differenze.