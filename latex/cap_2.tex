\chapter{Obiettivi}

	\section{Stato dell'arte}
	Il codice originale di partenza \`{e} strutturato in maniera tale che in una prima fase vengano letti i dati relativi all'immagine e al poligono e salvati rispettivamente in un array bidimensionale e in una struct contentente due array mono dimensionali (vertici e condizioni di bordo). 

	Viene, poi, eseguito un controllo per ogni punto della matrice per stabilire se questo sia interno o esterno al poligono e, in seguito, per tutte le celle vengono calcolate le condizioni di bordo. \\
	Una volta terminate le fasi di caricamento e di calcolo delle informazioni necessarie, viene generata la multi risoluzione sull'intera mappa a partire da un set di seed points letti in input da file.

	Il problema legato a questo tipo di approccio sta nel fatto che la matrice a multi risoluzione viene creata sulla base delle dimensioni della matrice che rappresenta la mappa altimetrica. \\
	Anche se le dimensioni della matrice a multi risoluzione sono inferiori fra le 10 e le 50 volte rispetto a quelle dell'immagine, la memoria da allocare rappresenta un collo di bottiglia per il programma che, in seguito, sfrutta la matrice a multi risoluzione per la simulazione del corso d'acqua.
	\section{Obiettivo della tesi}
	L'obiettivo di questa tesi \`{e}, dunque, quello di modificare il codice sorgente adottanto un approccio diverso in modo tale da creare la matrice a multi risoluzione prima del caricamento delle tavolette, e progettare l'algoritmo cos\`{i} da \`{i}occupare meno spazio in memoria per la memorizzazione della matrice a multi risoluzione. 

	\section{Soluzioni adottate}
	La soluzione adottata per ovviare al problema dell'allocazione di memoria \`{e} realizzata cos\`{i} da caricare inizialmente la struttura della matrice di tavolette e gli array del poligono e individuare la sottomatrice (bounging box) in cui questo ricade, sfruttando gli indici della matrice, calcolabili attraverso le coordinate geografiche (vengono lette solo le intestazioni delle tavolette!).\\ 
	Operando sul bounding box e partendo, poi, da un insieme di punti che rappresentano il perimetro del poligono, viene creata la matrice a multi risoluzione, caricando le altezze di tutti i punti delle tavolette solo alla fine, riducendo cos\`{i} lo spazio di memoria d'allocazione inizializzando, in una prima fase iniziale, tutti i valori delle celle della matrice ad un valore di default.
	Quando, in seguito, vengono lette le informazioni delle tavolette, il problema a cui bisogna ovviare \`{e} quello di caricare i valori delle altezze del terreno nelle corrette celle della matrice a multi risoluzione, che viene risolto attraverso opportuni calcoli ed opportune strutture dati di supporto.


