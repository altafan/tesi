\clearpage{\pagestyle{empty}\cleardoublepage}
\chapter*{Introduzione}
\addcontentsline{toc}{chapter}{Introduzione}

Negli utimi anni in Europa sono sempre maggiori i danni legati ad eventi di inondazione e nonostante sia stato stimato che nel 2050 la popolazione colpita da queste calamit� varier� tra le 500.000 e le 640.000 unit�, con un impatto economico tra i 20-40 mld di euro, sono ancora pi� allarmanti i dati derivanti dalle proiezioni del 2080 che mostrano un aumento di vittime (540.000-950.000 all'anno) e un conseguente aumento del derivante danno economico (30-100 mld di euro)\cite{articolo1}.\\
Dopo le nuove Direttive Europee, molti paesi hanno istituito programmi per l'analisi e la valutazione del rischio idrogeologico e in questo tipo di programmi la modellazione e la simulazione numerica dei flussi d'acqua rimane uno degli strumenti principalmente utilizzati\cite{gov}.

Il dipartimento di Matematica e Informatica in collaborazione col dipartimento di Ingegneria Civile e Ambientale e Architettura ha sviluppato un programma per la simulazione dei flussi d'acqua su aree di interesse variabili\cite{articolo2}. 
L'applicativo, come � facile pensare, richiede in input mappe geografiche di grandi dimensioni (di intere province ad esempio) e, data la grossa mole di informazione che queste mappe contenogono, � stato necessario introdurre nuove strutture dati in grado di rappresentarle in multi risoluzione.

Questa tesi si focalizza sulla fase di pre-proccessing in cui, data una mappa geografica di grandi dimensioni, viene eseguito un processo che restituisce l'immagine stessa in multi risoluzione, rendendo cos� possibile la valutazione di effeti su scala ridotta rispetto al grande dominio dell'intera immagine. \\
In particolare bisogna precisare che questa fase era gi� stata implementata e che mi sono occupato di modificarla per renderla pi� efficiente.

\`{E} necessario introdurre anche che il programma che compie le simulazioni di corsi d'acqua \`{e} implementato con codice che sfrutta la parallelizzazione del calcolo in pi\`{u} unit\`{a} di elaborazione che risiedono in componenti specifici chiamati GPU (Graphic Processing Unit). 

Il capitolo 1 di questo documento contiene le nozioni preliminari utili a comprendere la struttura e il funzionamento del progetto. Il capitolo 2 descrive brevemente lo stato dell'arte, l'obiettivo della tesi e il nuovo approccio, mentre nel capitolo 3 vengono illustrate nello specifico le nuove soluzioni adottate e come queste portano ad una maggiore efficienza. Infine nel capitolo 4 sono riportati i risultati e alcuni test cases e nel capitolo 5 le conclusioni e i possibili sviluppi futuri.
\clearpage{\pagestyle{empty}\cleardoublepage}